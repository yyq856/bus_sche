% !TEX program = pdflatex
\documentclass[11pt,a4paper]{article}
\usepackage[UTF8]{ctex}
\usepackage{amsmath, amssymb, amsthm}
\usepackage{geometry}
\usepackage{hyperref}
\geometry{margin=1in}

\title{集成双层公交优化的列生成方法说明}
\author{项目文档}
\date{\today}

\begin{document}
\maketitle

\section*{概述}
本文档面向 Overleaf 使用,说明我们在集成双层(系统成本 + 用户均衡)模型中的列生成(Column Generation, CG)流程。核心思想:
\begin{itemize}
  \item 主问题(Restricted Master Problem, RMP)是完整 MILP 的线性松弛,仅在当前列集合 $\Omega$ 上求解;
  \item 子问题(Pricing)利用主问题对偶变量,搜索具有最大\emph{正}“增益”的线路列(即具有正的“约化利润/降低成本”);
  \item 迭代添加新列,直至没有正增益列,再转 Phase II 在固定列集上求解整数化(得到上界)。
\end{itemize}

\section*{集合与记号}
\begin{itemize}
  \item $\mathcal{O},\mathcal{D}$:起讫点集合;$\mathcal{H}$:人群细分集合;
  \item $\mathcal{A}$:路段集合(有向边);$\Omega$:当前可用的公交线路列集合(每一列对应一条可运营的候选线路/路径,含频率档信息)。
  \item $D_{od,h}$:$(o,d,h)$ 需求;$\delta_{ap}\in\{0,1\}$:列 $p$ 是否经过路段 $a$;
  \item 成本:$F$ 为每条线路的固定成本,$c_p^{op}\ge 0$ 为列 $p$ 的可变运营成本(基于里程/频率),$c^{bg}(\cdot)$ 为背景交通的(Beckmann)成本,$c^{user}(\cdot)$ 为用户的广义时间成本;
  \item 主问题约束的对偶变量:$\gamma_{od,h}$(需求/守恒相关),$\rho_a$(路段流量相关)。
\end{itemize}

\section*{主问题(RMP:LP 松弛)}
记决策变量 $x_p\ge 0$ 表示选择列 $p\in\Omega$ 的“强度”(Phase I 为连续松弛;Phase II 再转整数化)。RMP 的目标是最小化系统总成本(运营 + 用户 + 背景),其中与列直接相关的部分写为 $F + c_p^{op}$:
\begin{align}
  \min_{x,\,\text{others}}\; Z_{\text{RMP}} 
  &= \sum_{p\in\Omega} (F + c_p^{op})\, x_p 
     + c^{user}(\cdot) + c^{bg}(\cdot). \label{eq:rmp_obj}
\end{align}

为与定价(pricing)联动,给出两类关键约束(其余与 SUE 外逼近相关的凸约束此处省略):
\begin{align}
  \text{(需求/守恒)}\quad 
  &\sum_{p\in\Omega: p\ \text{服务}\ (o,d)} x_p \;\; \text{与}\; D_{od,h}\; \text{满足守恒/分配关系}, &&\forall\, (o,d)\in\mathcal{O}\times\mathcal{D},\; h\in\mathcal{H}, \label{eq:rmp_dem}\[4pt]
  \text{(路段流量一致)}\quad 
  &v_a^{bg} + \sum_{p\in\Omega} \delta_{ap}\, x_p \;\; \text{与容量/流量定义一致}, &&\forall\, a\in\mathcal{A}. \label{eq:rmp_link}
\end{align}

对上述两类约束分别记对偶变量为 $\gamma_{od,h}$ 与 $\rho_a$。RMP 在当前 $\Omega$ 下达到最优时,我们从求解器中读取这些对偶值,交给定价子问题使用。

\paragraph{注} 实现中 RMP 是完整 MILP 的 LP 松弛版本(去除了整数与 KKT 对偶块),并包含 Beckmann 与熵项的外逼近切平面;具体线性化细节在工程代码中实现,但不影响列生成的结构化描述。

\section*{子问题(Pricing):最优增益线路搜索}
定价问题在给定对偶 $(\gamma,\rho)$ 后,寻找一条新线路(记为路径 $p$)使得其“增益”最大。我们使用如下的“增益/约化利润”定义(与实现一致,采用“正增益则加入”的判据):
\begin{align}
  \Delta(p; o,d,h) 
  &= \underbrace{\gamma_{od,h}}_{\text{需求对偶}}
     - \underbrace{\Big(F + c_p^{op}\Big)}_{\text{固定+运营成本}}
     - \underbrace{\sum_{a\in p} \rho_a}_{\text{路段对偶影响}}. \label{eq:reduced_cost}
\end{align}
当存在 $\Delta(p; o,d,h) > 0$ 的线路列时,将其加入 $\Omega$。如果对所有 $(o,d,h)$ 与候选路径 $p$ 都有 $\Delta\le 0$,则定价终止。

在实现上,我们令路径搜索的边权仅取非负的\emph{运营成本权重}(例如 $w_a\ge 0$,基于里程×单位成本),使用最短路/\emph{k}-最短路方法生成候选路径 $p$;随后再根据式\eqref{eq:reduced_cost} 叠加对偶项 $\sum_a \rho_a$ 与固定成本 $F$,计算真正的 $\Delta$。这样可避免负权导致的算法与库函数不稳定问题。

\section*{列生成算法(Phase I)}
\noindent 算法骨架如下:
\begin{enumerate}
  \item 初始化列集 $\Omega\leftarrow\Omega_0$(可用空集或少量启发式线路)。
  \item \textbf{主问题求解}:在 $\Omega$ 上求解 RMP(LP 松弛)得到最优值 $Z_{\text{RMP}}$,并提取对偶 $\gamma,\rho$。
  \item \textbf{定价子问题}:对每个 $(o,d,h)$,用非负运营权重在图上做最短路搜索,得到候选路径 $p$,计算 $\Delta(p; o,d,h)$。可一次加入前 $k$ 条正增益路径以加速收敛。
  \item 若所有候选的 $\Delta\le 0$,\textbf{停止}(Phase I 收敛);否则,将所有 $\Delta>0$ 的列加入 $\Omega$,返回第 2 步。
\end{enumerate}

\section*{Phase II:固定列集上的整数求解}
当 Phase I 无正增益列可加入时,当前 $Z_{\text{RMP}}$ 给出下界。此时在固定的 $\Omega$ 上恢复整数性(例如线路启用/频率档的 0-1/整数变量),求解 MILP 获得可行解与上界 $Z_{\text{MILP}}$,并计算间隙。若需要可继续用分支定界或切割加强。

\section*{实现与参数要点}
\begin{itemize}
  \item \textbf{对偶提取:} 我们以约束命名(如 demand\_elastic\_\{o\}\_\{d\}\_\{h\}、link\_flow\_(u,v))从求解器读取 $\gamma_{od,h},\rho_a$,避免 $O(n^2)$ 解析开销。
  \item \textbf{定价权重:} 路径搜索仅使用非负运营权重,\emph{对偶只参与式\eqref{eq:reduced_cost} 的增益计算},以保证图算法稳定。
  \item \textbf{成本标度:} 固定成本 $F$ 与单位运营成本需与对偶量纲匹配;实践中调参(如将 $F=10$)后,能稳定产生正增益列并加速收敛。
  \item \textbf{日志与监控:} 记录每次迭代的模型规模、RMP 目标、对偶稀疏度、候选列的 $\Delta$ 统计以及新增列数量,有助于诊断与调参。
\end{itemize}

\section*{小结}
以上给出了我们在集成双层公交优化中使用列生成的结构化做法:RMP 为完整模型的 LP 松弛,定价子问题用最短路搜索新线路并以对偶评估其增益,循环至无正增益列后进行 Phase II 整数化求解。该流程在工程实现中与 Gurobi 对偶读出、非负权重定价与现实参数标度保持一致。

\end{document}
